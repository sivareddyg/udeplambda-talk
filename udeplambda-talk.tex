\documentclass[mathserif,12pt]{beamer}
 
\mode<presentation>
{
%\usetheme{CambridgeUS}
\usetheme{Boadilla}
%\usecolortheme{whale}
%\usecolortheme{seahorse} % light blue 
%\usecolortheme{whale} % dark blue
%\usecolortheme{dolphin} %blue and black
\beamertemplatenavigationsymbolsempty
%\setbeamercovered{transparent}
\setbeamercovered{invisible}
\setbeamertemplate{itemize items}[default]
\setbeamertemplate{enumerate items}[default]
%\setbeamertemplate{footline}[frame number]{}
\setbeamertemplate{footline}{%
  \raisebox{5pt}{\makebox[\paperwidth]{\hfill\makebox[10pt]{\scriptsize\insertframenumber}}}}
}

\newcommand\blfootnote[1]{%
  \begingroup
  \renewcommand\thefootnote{}\footnote{#1}%
  \addtocounter{footnote}{-1}%
  \endgroup
}

%\setbeamercovered{dynamic}

%\usepackage[english]{babel}
%\usepackage[latin1]{inputenc}
\usepackage{tikz}
% \usetikzlibrary{fadings,shapes.arrows,shadows,arrows} 
% \usetikzlibrary{decorations.text}
\usepackage{xcolor}
\definecolor{brown}{RGB}{123,50,148}
\definecolor{darkgreen}{RGB}{0,100,0}

% \usepackage{dcolumn}
% \usepackage{transparent}
% font definitions, try \usepackage{ae} instead of the following
% three lines if you don't like this look
\usepackage{mathptmx}
\usepackage[scaled=.90]{helvet}

% \usepackage{courier}
% \usepackage{latexsym}
% \usepackage{multirow} 
% \usepackage{fancybox}
\usepackage{xspace}
%\usepackage{color}
% \usepackage{wrapfig}
% \usepackage{graphicx}
% \usepackage{epsfig}
% \usepackage{fancyhdr}
% \usepackage{natbib}
% \bibliographystyle{apalike}
% \usepackage[T1]{fontenc}
% \usepackage{amsmath}
% \usepackage{mathtools}
% \usepackage{subfloat}
% \usepackage{subfig}
% \usepackage{rotating}
% \usepackage[export]{adjustbox}
\usepackage{booktabs}

\usepackage{algorithmic}

\usepackage{xkeyval}
\usepackage{xargs} % commandx 
\usepackage{todonotes}
%\usepackage{marginnote}
\usepackage{xspace}
\presetkeys{todonotes}{inline}{}
%\newcommandx{\todoir}[2][1=]{\todo[inline]{SR: #2}\xspace}
%\newcommandx{\todor}[2][1=]{\todo[linecolor=red,backgroundcolor=red!25,bordercolor=red,#1]{SR: #2}\xspace}

\usepackage{pifont} % for checkmark and xmark
\newcommand{\cmark}{\ding{51}}%
\newcommand{\xmark}{\ding{55}}%

%\usepackage[labelformat=empty]{caption}
\setbeamertemplate{caption}{\raggedright\insertcaption\par}

% \usepackage{graphics}
\usepackage{amsbsy,amsfonts}

%\usepackage{lmodern}
%\usefonttheme[onlymath]{serif}
%\newcommand{\etal}{\textit{et al.\ }}

\newcommand \ignore[1]{}
\newcommand \siva[1]{} 
\newcommand \denot[2]{[\![#1]\!]_\mathit{#2}}
%\newcommand{\argmax}[1]{\arg\underset{#1}{\max}\;}
\newcommand{\argmax}{\operatornamewithlimits{arg\,max}}
\newcommand \kcaz{\textsc{kcaz13}\xspace}
\newcommand \psempre{\textsc{ParaSempre}\xspace}
\newcommand \mwg{\textsc{mwg}\xspace}
\newcommand{\word}[1]{\textsl{#1}}        % mention of word
\newcommand{\hlight}[1]{{\color{blue!80} #1}}

\newcommand \gp{\textsc{GraphParser}\xspace}
\newcommand \free{Free917\xspace}
\newcommand \webq{WebQuestions\xspace}
\newcommand \deplambda{\textsc{DepLambda}\xspace}
\newcommand \deptree{\textsc{DepTree}\xspace}
\newcommand \simplegraph{\textsc{SimpleGraph}\xspace}
\newcommand \ccggraph{\textsc{CCGGraph}\xspace}
\newcommand \deplambdap{DepLambda+\xspace}
\newcommand \clue{\textsc{ClueWeb09}\xspace}
\newcommand \contract{\textsc{contract}\xspace}
\newcommand \expand{\textsc{expand}\xspace}


\newcommand\vect[1]{\overrightarrow{\mathbf{#1}}}
\newcommand\type[1]{\textbf{\textsc{type}}[#1]\xspace}

\renewcommand{\l}{\lambda}
\newcommand{\lx}{\lambda x }
\newcommand{\ly}{\lambda y }
\newcommand{\lz}{\lambda z }
\newcommand{\lxy}{\lx \ly}
\newcommand{\lf}{\lambda f .}
\renewcommand{\lg}{\lambda g .}
\newcommand{\fo}{f_1}
\newcommand{\ft}{f_2}
\newcommand{\ftt}{f_3}
\newcommand{\lfo}{\lambda \fo }
\newcommand{\lft}{\lambda \ft }
\newcommand{\lftt}{\lambda \ftt }
\renewcommand{\le}{\lambda e }
\newcommand{\und}{$\_$}

\newcommand{\ex}{\exists x }
\newcommand{\ey}{\exists y }
\newcommand{\ez}{\exists z }
\newcommand{\exy}{\lx \ly}
\newcommand{\ef}{\exists f }
\newcommand{\efo}{\exists \fo }
\newcommand{\et}{\exists \ft }
\newcommand{\eftt}{\exists \ftt }

\newcommand{\itype}{{\bf Ind}}
\newcommand{\etype}{{\bf Event}}
\newcommand{\ptype}{\itype \times \etype}
\newcommand{\ftype}{\eta}
\newcommand{\btype}{{\bf Bool}}
\renewcommand{\land}{\wedge}
\newcommand{\true}{\lambda x \lspace \textsc{true}}
\newcommand{\lspace}{.\,}
\newcommand{\bind}{\textsc{bind}}
\newcommand{\eqop}{\textsc{eq}}

\title[\tiny Universal Semantic Parsing]{Universal Semantic Parsing}

\vspace{-2cm}
\author[]{\small Siva Reddy\\
Stanford University}

\vspace{1cm}
\institute{
\begin{center}
\begin{tabular}{c c c c}
\includegraphics[height=2cm]{figures/oscar} &
\includegraphics[height=2cm]{figures/petrov} &
\includegraphics[height=2cm]{figures/steedman} &
\includegraphics[height=2cm]{figures/mirella} \\
Oscar~T\"ackstr\"om & Slav~Petrov  & Mark~Steedman & Mirella~Lapata \\
\end{tabular}
\end{center}

\hlight{Google} and \hlight{University of Edinburgh}\\
}
  
% \date{}
\date[]{}

\begin{document}

\begin{frame}
\titlepage

% \begin{center}
% NAACL 2016
% \end{center}
\end{frame}

\section{Introduction}

\ignore{
\begin{frame}
\frametitle{Dependency Trees help Semantics}
\centering
\includegraphics[trim=15em 18em 14em 12em,clip=true,scale=0.5]{figures/prep-ambiguity}

\vspace{2em}
\includegraphics[trim=1em 9em 21em 0em,clip=true,scale=1.2]{figures/pp-ambiguity}
\end{frame}
}

\begin{frame}
\frametitle{Dependency Trees help Semantics}
\begin{center}
\only<1>{\vspace{-1em} \includegraphics[trim=1em 5em 24em 5.1em,clip=true,scale=1.3]{figures/telugu-dependency-example} \\}
  \only<2>{\vspace{5em}\includegraphics[trim=1em 5em 24em 5.1em,clip=true,scale=1.3]{figures/telugu-dependency-example} \\}
\only<3,4>{\includegraphics[trim=1em 5em 24em 0em,clip=true,scale=1.3]{figures/telugu-dependency-example} \\}
\only<2,3>{\includegraphics[scale=0.45]{figures/monkey-eat-banana}}
\only<4>{\includegraphics[scale=0.26]{figures/banana-eat-monkey}}
\end{center}
\end{frame}

\begin{frame}
\frametitle{Universal Dependencies}
%\large Homogeneous syntactic representation across languages
\begin{center}
\includegraphics[trim=0em 5em 15em 0em,clip=true,scale=1.2
]{figures/dependency-word-order-english}

\only<1>{\vspace{8em}}
\only<2>{\vspace{4.8em}\includegraphics[trim=0em 3.5em 14em 6.5em,clip=true,scale=1.2
  ]{figures/dependency-word-order-telugu}}
\only<3>{\includegraphics[trim=0em 3.5em 14em 0.5em,clip=true,scale=1.2
  ]{figures/dependency-word-order-telugu}
  \vspace{-0.5em}}
\end{center}
\end{frame} 

\begin{frame}
\frametitle{Universal Dependencies}
\large

Common syntactic representation in 50+ languages

\vspace{2em}

Manning laws: 
\begin{itemize}
\item Satisfactory linguistic analysis

\vspace{1em}
\item Easy to comprehend (e.g., 40 labels)

\vspace{1em}
\item Rapid and consistent annotations

\vspace{1em}
\item High accuracy parsing {\footnotesize [Dozat et al. 2017]}
\end{itemize}
\end{frame}

\begin{frame}
\frametitle{Dependency Tree to Semantics}
\centering

\vspace{1em}

\includegraphics[trim=26em 0em 3.8em 2em,clip=true,scale=0.4]{figures/syntax-to-semantics-cartoon}

\vspace{2em}

\Large
 Dependencies \alert{lack} a formal theory of semantics
\end{frame}

\begin{frame}
 \frametitle{This Talk}
 \Large \centering

Universal Semantic Parsing:\\
Language-agnostic conversion of \\Universal Dependencies to Logical Forms
 
\end{frame}

\begin{frame}
 \frametitle{This Talk: Contributions}
 \large 

Universal Dependencies to \textbf{general-purpose} logical forms

\vspace{1em}
A general solution that also works for \textbf{Dependency Graphs}

\vspace{1em}
Multilingual evaluation of logical forms on \textbf{Freebase QA}

\vspace{1em}
WebQuestions and GraphQuestions QA datasets in \textbf{German} and \textbf{Spanish}
 
\end{frame}


\begin{frame}
\frametitle{Dependency Tree to Semantics}
\large
\only<1>{
  \hlight{Principle of Compositionality:} the semantics of a complex expression is determined by the semantics of its constituent expressions and the rules used to combine them}
\only<2>{
  \hlight{Principle of Compositionality:} the semantics of a \alert{complex expression} is determined by the semantics of its constituent expressions and the rules used to combine them}
\only<3>{
  \hlight{Principle of Compositionality:} the semantics of a \alert{complex expression} is determined by the semantics of its {\color{darkgreen}
 constituent expressions} and the rules used to combine them}
\only<4>{
  \hlight{Principle of Compositionality:} the semantics of a \alert{complex expression} is determined by the semantics of its {\color{darkgreen}
    constituent expressions} and the {\color{purple} rules} used to combine them}
  
  \visible<2->{
  \vspace{2em}
  \alert{Complex expression} is the dependency tree}
  
\visible<3->{
  \vspace{2em}
  {\color{darkgreen} Constituent expressions} are subtrees}
  

\visible<4->{
  \vspace{2em}
  {\color{purple} Rules} are the dependency labels
}
\end{frame}

%\begin{frame}
%\Large
%\centering
%\vspace{1.5em}
%Universal Semantic Parsing
%\end{frame}
%
%\begin{frame}
%\frametitle{Goals}
%\Large
%\begin{enumerate}
%\item Dependencies to logical forms
%\vspace{1em}
%\item Compositional
%\vspace{1em}
%\item Language-agnostic conversion
%\begin{itemize}
%  \large
%  \vspace{1em}
%  \item Dependency labels and postags dictate the semantics
%  \vspace{1em}
%  \item No token-specific semantics
%\end{itemize}
%\end{enumerate}
%\end{frame}

\begin{frame}
\frametitle{Universal Semantic Parsing: Objectives}
\large
Logical form must be built

\vspace{1em}
\begin{enumerate}
\item \textbf{compositionally} from the dependency tree 
  \vspace{2em}
\item in a \textbf{language-agnostic} manner
\begin{itemize}
  \item \large Dependency labels and postags dictate the semantics, \textbf{not} the words
\end{itemize}
\end{enumerate}



\end{frame}

\ignore{
\begin{frame}
\frametitle{Compositional}
\vspace{-1em}
\begin{center}
\only<1>{\vspace{-5em}\includegraphics[trim=2em 5em 12em 0em,clip=true,scale=1.0]{figures/dependency-reduced-relative-ud}}
\only<2>{\vspace{2em}\includegraphics[trim=2em 2em 12em 0em,clip=true,scale=1.0]{figures/dependency-reduced-relative-ud-parg}}
\only<3,4>{\includegraphics[trim=2em 5em 12em 0em,clip=true,scale=1.0]{figures/dependency-reduced-relative-ud-box1}}
\only<5,6>{\includegraphics[trim=2em 5em 12em 0em,clip=true,scale=1.0]{figures/dependency-reduced-relative-ud-box2}}
\only<7>{\includegraphics[trim=2em 5em 12em 0em,clip=true,scale=1.0]{figures/dependency-reduced-relative-ud-box3}}
\only<8->{\includegraphics[trim=1.5em 5em 12em 0em,clip=true,scale=1.0]{figures/dependency-reduced-relative-ud-box4}}
 
\only<3>{\includegraphics[trim=29.5em 51.5em 4.5em 0em,clip=true,scale=0.9]{figures/dependency-reduced-relative-derivation-ud}\vspace{6.5em}}
\only<4>{\includegraphics[trim=29.5em 51.5em 4.5em 0em,clip=true,scale=0.9]{figures/dependency-reduced-relative-derivation-ud}\\
\includegraphics[trim=30.6em 45em 6em 3em,clip=true,scale=0.9]{figures/dependency-reduced-relative-derivation-ud}\vspace{2em}}
\only<5>{\includegraphics[trim=22em 51.5em 9em 0em,clip=true,scale=0.9]{figures/dependency-reduced-relative-derivation1-ud}\vspace{6.5em}} 
\only<6>{\includegraphics[trim=22em 51.5em 9em 0em,clip=true,scale=0.9]{figures/dependency-reduced-relative-derivation1-ud}\\
\includegraphics[trim=23em 45em 9em 5em,clip=true,scale=0.9]{figures/dependency-reduced-relative-derivation1-ud}\vspace{4em}}
\only<7>{\includegraphics[trim=9em 51.3em 9em 0em,clip=true,scale=0.9]{figures/dependency-reduced-relative-derivation2-ud}\\
\includegraphics[trim=9em 45em 9em 5.5em,clip=true,scale=0.9]{figures/dependency-reduced-relative-derivation2-ud}\vspace{4em}} 

\only<2>{\vspace{2cm}$\lambda x. \exists yz \lspace \mathrm{located}(z_e) \wedge  \mathrm{Pixar}(x_a) \wedge \mathrm{CA}(y_a) \wedge$ \\ $\mathrm{\color{darkgreen}company}(x_a) \wedge \mathrm{\color{red}arg_2}(z_e, x_a) \wedge \mathrm{\color{blue} arg_{in}}(z_e, y_a)$}
\only<8>{\vspace{2cm}$\lambda x. \exists yz \lspace \mathrm{located}(z_e) \wedge  \mathrm{Pixar}(x_a) \wedge \mathrm{CA}(y_a) \wedge$ \\ $\mathrm{company}(x_a) \wedge \mathrm{arg_2}(z_e, x_a) \wedge \mathrm{arg_{in}}(z_e, y_a)$}
\end{center} 
\end{frame}
}

\begin{frame}
\frametitle{Compositional}
% \framesubtitle{Binarization}
\begin{center}
\vspace{1em}
\includegraphics[trim=2em 9.4em 10em 0em,clip=true,scale=1.3]{figures/pixar}

\vspace{4cm}

\includegraphics[trim=7em 10em 7em 2.5em,clip=true,scale=1.1]{figures/dependency-transitive-derivation_full}
\vspace{-2cm}
\end{center}
\end{frame}

\begin{frame}
\frametitle{Compositional}
% \framesubtitle{Binarization}
\begin{center}
\vspace{-5em}
% TODO: remove types on dependency labels.
\includegraphics[trim=2em 9.4em 10em 0em,clip=true,scale=1.3]{figures/pixar}

\vspace{2cm}
\begin{block}{}
\centering
Dependency labels drive the composition
\end{block}
\end{center}
\end{frame}

\begin{frame}[noframenumbering]
\frametitle{Compositional}
% \framesubtitle{Binarization}
\vspace{-2.5em}
\centering

\only<1>{\includegraphics[trim=2em 9.4em 10em 0em,clip=true,scale=1.3]{figures/pixar_root}}
\only<2->{\includegraphics[trim=2em 9.4em 10em 0em,clip=true,scale=1.3]{figures/pixar_dobj}}

\tikz[remember picture] \node[coordinate] (n1) {};\\
\vspace{2cm}
\tikz[remember picture] \node[coordinate] (n2) {};

\begin{tikzpicture}[remember picture,overlay]   %% use here too
  \draw [->,>=stealth,shorten >=1pt,blue!20!white,line width=10pt] (n1) -- node[xshift=6em,black] {\footnotesize $\ldots >$ dobj $> \dots >$  nsubj $> \ldots$ } (n2);
  \draw [->,>=stealth,blue!20!white,sloped,line width=10pt] (n1) -- node[black] {\footnotesize } (n2);
\end{tikzpicture}

\visible<2->{({\color{red} dobj} {\color{blue} acquired} {\color{blue!40!green!60!black} Pixar})}
\end{frame}

\begin{frame}[noframenumbering]
\frametitle{Compositional}
% \framesubtitle{Binarization}
\vspace{-2.5em}
\centering

\includegraphics[trim=2em 9.4em 10em 0em,clip=true,scale=1.3]{figures/pixar_dobj_phrase}

\tikz[remember picture] \node[coordinate] (n1) {};\\
\vspace{2cm}
\tikz[remember picture] \node[coordinate] (n2) {};

\begin{tikzpicture}[remember picture,overlay]   %% use here too
  \draw [->,>=stealth,shorten >=1pt,blue!20!white,line width=10pt] (n1) -- node[xshift=6em,black] {\footnotesize $\ldots >$ dobj $> \dots >$  nsubj $> \ldots$ } (n2);
  \draw [->,>=stealth,blue!20!white,sloped,line width=10pt] (n1) -- node[black] {\footnotesize } (n2);
\end{tikzpicture}

{\color{blue} (dobj acquired Pixar)}
\end{frame}

\begin{frame}[noframenumbering]
\frametitle{Compositional}
% \framesubtitle{Binarization}
\vspace{-2.5em}
\centering

\only<1>{\includegraphics[trim=2em 9.4em 10em 0em,clip=true,scale=1.3]{figures/pixar_nsubj}}
\only<2->{\includegraphics[trim=2em 9.4em 10em 0em,clip=true,scale=1.3]{figures/pixar_nsubj_phrase}}

\tikz[remember picture] \node[coordinate] (n1) {};\\
\vspace{2cm}
\tikz[remember picture] \node[coordinate] (n2) {};

\begin{tikzpicture}[remember picture,overlay]   %% use here too
  \draw [->,>=stealth,shorten >=1pt,blue!20!white,line width=10pt] (n1) -- node[xshift=6em,black] {\footnotesize $\ldots >$ dobj $> \dots >$  nsubj $> \ldots$ } (n2);
  \draw [->,>=stealth,blue!20!white,sloped,line width=10pt] (n1) -- node[black] {\footnotesize } (n2);
\end{tikzpicture}

\only<1>{({\color{red} nsubj} {\color{blue} (dobj acquired Pixar)} {\color{blue!40!green!60!black} Disney})}
\only<2->{\color{blue}(nsubj (dobj acquired Pixar) Disney)}
\end{frame}

\begin{frame}[noframenumbering]
\frametitle{Compositional}
% \framesubtitle{Binarization}
\begin{center}
%\vspace{-1em}
\includegraphics[trim=2em 9.4em 10em 0em,clip=true,scale=1.3]{figures/pixar}

\vspace{3em}
(nsubj (dobj acquired Pixar) Disney)
\vspace{3em}

\visible<1->{
\includegraphics[trim=7em 0em 7em 2.5em,clip=true,scale=1.1]{figures/dependency-transitive-derivation_full}
\vspace{-1.5cm}}
\end{center}
\end{frame}

\ignore{
\begin{frame}
\frametitle{Language-agnostic Conversion}
\framesubtitle{Substitution}
\vspace{-3.3em}
\begin{center}
\includegraphics[trim=2em 9.4em 10em 0em,clip=true,scale=1.3]{figures/pixar_dobj}

\end{center}

\vspace{1cm}

\begin{block}{Lambda Calculus Basic Types}
\begin{itemize}
  \item Individuals: $\itype$ (also denoted by $._a$)
  \item Events: $\etype$ (also denoted by $._e$)
  \item Truth values: $\btype$
\end{itemize}
\end{block}
\end{frame}
}

\begin{frame}
\frametitle{Language-agnostic Conversion}
% \framesubtitle{Substitution}
\vspace{-2.4em}
\begin{center}
\includegraphics[trim=2em 9.4em 10em 0em,clip=true,scale=1.3]{figures/pixar_dobj}

\end{center}

\vspace{1cm}

\begin{block}{\centering Lambda Expression for words}
\vspace{-0.5cm}
\begin{align*}
  VERB \Rightarrow \lambda x \lspace \hbox{word}(x_e) \\
  PROPN \Rightarrow \lambda x \lspace \hbox{word}(x_a) 
\end{align*}
\vspace{-0.5cm}
\end{block}
\end{frame}

\begin{frame}[noframenumbering]
\frametitle{Language-agnostic Conversion}
% \framesubtitle{Substitution}
\vspace{-2.4em}
\begin{center}
\includegraphics[trim=2em 9.4em 10em 0em,clip=true,scale=1.3]{figures/pixar_dobj}

\end{center}

\vspace{1cm}

\begin{block}{\centering Lambda Expression for words}
\vspace{-0.5cm}
\begin{align*}
  \hbox{\color{blue} acquired} & \Rightarrow  \lambda x \lspace \hbox{acquired}(x_e)  \\
  \hbox{\color{blue!40!green!60!black} Pixar} & \Rightarrow  \lambda x \lspace \hbox{Pixar}(x_a) 
\end{align*}
\vspace{-0.5cm}
\end{block}
\end{frame}

\begin{frame}
\frametitle{Language-agnostic Conversion}
% \framesubtitle{Substitution}
\vspace{-2.4em}
\begin{center}
\includegraphics[trim=2em 9.4em 10em 0em,clip=true,scale=1.3]{figures/pixar_dobj}

\end{center}

\vspace{1cm}

\begin{block}{\centering Lambda Expression for dependency labels}
\vspace{-0.5cm}
\begin{align*}
  \hbox{\alert{dobj}} & \Rightarrow  \lambda \hlight{\mathbf{f}}\;\; \lambda {\color{blue!40!green!60!black} \mathbf{g}}\;\; \lambda \hlight{\mathbf{z}}\; \lspace \; \exists {\color{blue!40!green!60!black} \mathbf{x}}\; \lspace \; \hlight{\mathbf{f(z)}}\;\;\; \land \;\;\; {\color{blue!40!green!60!black} \mathbf{g(x)}}  \;\;\; \land\;\;\; \alert{\mathbf{arg_2}}(\hlight{\mathbf{z_e}}, {\color{blue!40!green!60!black} \mathbf{x_a}})
\end{align*}
\vspace{-0.5cm}
\end{block}
\end{frame}

\begin{frame}
\frametitle{Language-agnostic Conversion}
% \framesubtitle{Substitution}
\vspace{2em}
\includegraphics[trim=1em 0em 0em 4em,clip=true,scale=1]{figures/dobj-lambda-calculus}
\end{frame} 

\subsection{DepLambda Composition}
\begin{frame}
\frametitle{Dependencies to Logical Forms}
\framesubtitle{Composition}
\begin{center}
\includegraphics[trim=2em 9.4em 10em 0em,clip=true,scale=1.3]{figures/pixar_dobj} 

\vspace{0.8cm}

\only<1>{\includegraphics[trim=7em 16em 7em 0em,clip=true,scale=1.1]{figures/dependency-transitive-derivation_dobj}\vspace{8em}}
\only<2>{\includegraphics[trim=7em 12em 7em 0em,clip=true,scale=1.1]{figures/dependency-transitive-derivation_dobj}\vspace{4em}}
\only<3>{\includegraphics[trim=7em 8em 7em 0em,clip=true,scale=1.1]{figures/dependency-transitive-derivation_dobj}}
\end{center}
\end{frame}

\begin{frame}
\frametitle{Dependencies to Logical Forms}
\framesubtitle{Composition}
\begin{center}
\only<1>{\includegraphics[trim=2em 9.4em 10em 0em,clip=true,scale=1.3]{figures/pixar_dobj_phrase}}
\only<2-4>{\includegraphics[trim=2em 9.4em 10em 0em,clip=true,scale=1.3]{figures/pixar_nsubj}}
\only<5->{\includegraphics[trim=2em 9.4em 10em 0em,clip=true,scale=1.3]{figures/pixar_nsubj_phrase}}

\vspace{0.8cm}

\only<1>{\hspace{1.2em}\includegraphics[trim=10em 16em 10em 0em,clip=true,scale=1.1]{figures/dependency-transitive-derivation-nsubj}\vspace{8em}}
\only<2>{\includegraphics[trim=3em 16em 3em 0em,clip=true,scale=1.1]{figures/dependency-transitive-derivation-nsubj}\vspace{8em}}
\only<3>{\includegraphics[trim=3em 12em 3em 0em,clip=true,scale=1.1]{figures/dependency-transitive-derivation-nsubj}\vspace{4em}}
\only<4>{\includegraphics[trim=3em 8em 3em 0em,clip=true,scale=1.1]{figures/dependency-transitive-derivation-nsubj}}
\only<5>{\includegraphics[trim=3em 8em 3em 0em,clip=true,scale=1.1]{figures/dependency-transitive-derivation_full}}
\end{center}
\end{frame}

\begin{frame}
\frametitle{In a nutshell}
\large

Dependency tree is a series of \hlight{compositions}

\vspace{2em}
Dependency label defines the \hlight{composition function}

\vspace{2em}
Each function takes two semantic \hlight{sub-expressions}

\vspace{2em}
Returns semantics of the \hlight{larger expression}
\end{frame}

\begin{frame}
\frametitle{Dependencies to Logical Forms}
 \vspace{0.6cm}
\begin{columns}
  \begin{column}{0.40\textwidth}
   \centering
  \vspace{-1em}
\includegraphics[trim=1.5em 9em 28em 0em,clip=true,scale=1.3]{figures/appos}   
  \end{column}
    \begin{column}{0.1\textwidth}
  \end{column}
  \begin{column}{0.58\textwidth}
    \large $\textsl{appos}  =$ \\ \hspace{0.5em} $ \lambda f \lambda g \lambda x \lspace f(x) \land g(x)$ \\
  \end{column}
 \end{columns}
 
 \pause
 \vspace{0.6cm}
 \begin{columns}
  \begin{column}{0.39\textwidth}
   \centering
\includegraphics[trim=1.5em 10em 26em 0em,clip=true,scale=1.2]{figures/amod-verb}
  \end{column}
  \begin{column}{0.1\textwidth}
  \end{column}
  \begin{column}{0.59\textwidth}
\large    $\textsl{amod}= $ \\ \hspace{0.5em} $\lambda f \lambda g \lambda x \lspace \exists z \lspace f(x) \land g(z) \land \hbox{amod}^i(z_e, x_a)$ 
  \end{column}
  
 \end{columns}
 
% \pause
%  \vspace{0.6cm}
% \begin{columns}
%  \begin{column}{0.39\textwidth}
%   \centering
%\includegraphics[trim=1em 9em 28em 0em,clip=true,scale=1.3]{figures/conj}
%  \end{column}
%  \begin{column}{0.1\textwidth}
%  \end{column}
%  \begin{column}{0.61\textwidth}
%\large    $\textsl{conj} =$ \\ $ \lambda f \lambda g \lambda z \lspace \exists x y \lspace f(x) \land g(y) \land \mathrm{coord}(z,x,y)$ 
%  \end{column}
% \end{columns}
\end{frame}

\begin{frame}
\frametitle{Dependencies to Logical Forms}
\vspace{-1em}
\begin{center}
\only<1>{\vspace{-5em}\includegraphics[trim=2em 5em 12em 0em,clip=true,scale=1.0]{figures/dependency-reduced-relative-ud}}
\only<2>{\vspace{2em}\includegraphics[trim=2em 2em 12em 0em,clip=true,scale=1.0]{figures/dependency-reduced-relative-ud-parg}}

\only<2>{\vspace{2cm}$\lambda x. \exists yz \lspace \mathrm{located}(z_e) \wedge  \mathrm{Pixar}(x_a) \wedge \mathrm{CA}(y_a) \wedge$ \\ $\mathrm{\color{darkgreen}company}(x_a) \wedge \mathrm{\color{red}arg_2}(z_e, x_a) \wedge \mathrm{\color{blue} arg_{in}}(z_e, y_a)$}
\end{center}
\end{frame}

\begin{frame}
\frametitle{UD labels are insufficient in few cases}
\large

UD may conflate different semantic phenomenon\\
\begin{itemize}
\item DET could mean a determiner or a question word e.g.,~\textsl{what} vs \textsl{the}
%\item Morphology classes are useful in these cases
\end{itemize}

\pause
\vspace{2em}
UD does not have long-distance dependencies \\ 
e.g., in control constructions

\pause
\vspace{2em}
\hlight{Solution:} \textbf{Enhancement step}, a lightweight preprocessing {\small [Schuster and Manning 2016]}
\end{frame}

\begin{frame}
\frametitle{Enhancement Step}
\framesubtitle{Question Words, Long-distance, Language-specific labels, Quantifiers}
\centering

\vspace{0.2cm} 
\includegraphics[trim=0em 1.7em 0em 0em,clip=true,scale=1.1]{figures/control-obj-extraction-crop} \\

\tikz[remember picture] \node[coordinate] (n1) {};\\
\vspace{2cm} 
\tikz[remember picture] \node[coordinate] (n2) {};

\begin{tikzpicture}[remember picture,overlay]   %% use here too
  \draw [->,>=stealth,shorten >=1pt,blue!20!white,line width=10pt] (n1) -- node[xshift=6em,black] { } (n2);
  \draw [->,>=stealth,blue!20!white,sloped,line width=10pt] (n1) -- node[black] {\footnotesize Enhancement} (n2);
\end{tikzpicture}

\includegraphics[trim=0em 0em 0em 0em,clip=true,scale=1.1]{figures/control-obj-extraction-crop}
\end{frame}

\begin{frame}
\frametitle{Dependency Graphs to Logical Forms}
\centering
\includegraphics[trim=0em 0em 0em 0em,clip=true,scale=1.1]{figures/control-obj-extraction-crop} \\

\tikz[remember picture] \node[coordinate] (n1) {};\\
\vspace{1cm} 
\tikz[remember picture] \node[coordinate] (n2) {};

\begin{tikzpicture}[remember picture,overlay]   %% use here too
  \draw [->,>=stealth,shorten >=1pt,blue!20!white,line width=10pt] (n1) -- node[xshift=6em,black] { } (n2);
  \draw [->,>=stealth,blue!20!white,sloped,line width=10pt] (n1) -- node[black] {} (n2);
\end{tikzpicture}

\includegraphics[trim=0em 0em 0em 0em,clip=true,scale=1.1]{figures/control-obj-extraction-bind-crop}
\end{frame}

\begin{frame}
\frametitle{Dependency Graphs to Logical Forms}
\centering
\includegraphics[trim=0em 0em 0em 0em,clip=true,scale=1.1]{figures/control-obj-extraction-bind-crop}

\vspace{2em}
\begin{block}{\center Lambda Expressions}
\begin{tabbing}
\textsc{bind} \hspace{1em} \=  = \hspace{1em} \= $\lambda f \lambda g \lambda x \lspace f(x) \land g(x)$ \\
$\mathrm{xcomp}$ \> = \hspace{1em} \> $\lambda f g x \lspace \exists y \lspace f(x) \wedge g(y) \wedge \mathrm{xcomp}(x_e, y_e)$ \\
$\Omega $ \> = \> $\lambda x. \textsc{eq}(x, \omega)$ 
\end{tabbing}
\end{block}
\end{frame} 


\begin{frame}
\Large
\centering
\vspace{1.5em}
Evaluation of logical forms on \\
Freebase Semantic Parsing
\end{frame}

\section{Semantic Parsing as Graph Matching}
\begin{frame}
\frametitle{Freebase Semantic Parsing}
\framesubtitle{[Berant et al., 2013, Kwiatkowski et al., 2013]}
\vspace{-1em}
\begin{columns}
 \begin{column}{0.55\textwidth}
 %\vspace{-1cm}
 \begin{figure}
\begin{tikzpicture}[
  font=\sffamily,
  every matrix/.style={ampersand replacement=\&,column sep=0.2cm,row sep=1cm},
  myboxNL/.style={draw,thick,rounded corners,text width=15em,inner sep=0.15cm,fill=green!10},
  myboxMR/.style={draw,rounded corners,text width=15em,inner 
  sep=0.05cm,fill=red!5,draw=red!10}, 
  myboxAns/.style={draw,thick,rounded
  corners,text width=15em,inner sep=0.15cm,fill=yellow!10}, myArrow/.style={->,>=stealth,shorten >=1pt,blue!20!white,sloped,line width=10pt},
  myArrow1/.style={->,>=stealth,shorten >=1pt,blue!20!white,above,sloped,line width=10pt},
  every node/.style={align=center}]
\matrix{
    \node[myboxNL] (NL) {\textbf{Question} \\ Who is the director of Titanic?}; \\ 
    \only<2->{\node[myboxMR] (MR) [text=black!16] {
    \textbf{Grounded Logical Form} \\ $\lambda x.\exists e.\;\mbox{film.director}(x) \wedge \mbox{film.directed\_by}(e) \wedge \mbox{arg2}(y,x) \wedge \mbox{arg1}(e,\mbox{Titanic})$ }; \\}
    \node[myboxAns] (Ans) {\textbf{Answer} \\ \{James Cameron\}}; \\
    };
\end{tikzpicture}

\only<2->{\begin{tikzpicture}[rounded corners]

 \node[text width=15em,fill=white!0,draw=white!0](q_A) at (-5,-5) [text=red!80]{
\vspace{-4.2cm} \\ \hspace{2cm} Latent
 };
\end{tikzpicture}}
\end{figure}
 \end{column}

 \begin{column}{0.45\textwidth}
 \vspace{-1cm}
 \begin{figure}
  \includegraphics[scale=0.45]{figures/titanic} 
 \end{figure}
 \end{column}
\end{columns}
\end{frame}

\begin{frame}
\frametitle{Freebase Semantic Parsing {\small [Reddy et al. 2014, 2016]}}
\only<1>{\includegraphics[scale=0.85]{figures/outline1.pdf}} 
\only<2>{\includegraphics[scale=0.85]{figures/outline2.pdf}}
\only<3>{\includegraphics[scale=0.85]{figures/outline3.pdf}}
\only<4>{\includegraphics[scale=0.85]{figures/outline.pdf}}
\end{frame}

\begin{frame}
\frametitle{Multilingual Freebase Semantic Parsing}

\includegraphics[scale=0.58]{figures/multilingual-grounding}

\end{frame}

%\begin{frame}
%\frametitle{Learning Model}
%\large
%
%\hlight{Structured Perceptron:} Ranks grounded and ungrounded graph pairs 
%\only<1>{\includegraphics{figures/perceptron1}}
%\only<2>{\includegraphics{figures/perceptron2}}
%\only<3>{\includegraphics{figures/perceptron3}}
%\only<4>{\includegraphics{figures/perceptron4}}
%\only<5->{\includegraphics{figures/perceptron5}}
%
%\vspace{0.5cm}
%
%
%
%\visible<6->{\hlight{Training:} Use gold graph to update weights 
%$$\theta \leftarrow \theta +
%\Phi(g^+,u^+,q,\mathit{KB})\hspace{-.3ex}-\hspace{-.3ex}\Phi(\hat{g},\hat{u},q,\mathit{KB})$$}
%\end{frame}

\begin{frame}
\frametitle{Experimental Setup}
\large
69 lambda calculus rules

\vspace{2em}
BiLSTM Parser {\small [Kipperwiser and Goldberg 2016]}\\
\begin{itemize}
\item English: 81.8 \\
\item German: 74.7 \\
\item Spanish: 82.2 \\
\end{itemize}
\end{frame}

\section{Evaluation} 

\begin{frame}
\frametitle{Multilingual WebQuestions and GraphQuestions}
\centering
\large
\begin{tabular}{l l}
\midrule
\multicolumn{2}{c}{WebQuestions} \\
\midrule
en & What language do the people in Ghana speak?\\
de & Welche Sprache wird in Ghana gesprochen?\\
es & ?`Cu\'{a}l es la lengua de Ghana?\\
\midrule
\multicolumn{2}{c}{GraphQuestions} \\
\midrule
en & NASA has how many launch sites? \\
de & Wie viele Abschussbasen besitzt NASA? \\
es & ?`Cu\'{a}ntos sitios de despegue tiene NASA? \\
\end{tabular}
\end{frame}

\begin{frame}
\large
\frametitle{Models}
\hlight{$\underset{\mathrm{bag\; of\; words}}{\simplegraph:}$ }All entities connected to a single event
%\begin{itemize}
 % \item Does not handle compositional questions% in addition to a \textsc{target} node.
%\end{itemize}

% \pause

%\vspace{2em} 
% \hlight{\ccggraph: } CCG logical forms 

%\pause
\vspace{2em}
\hlight{\deptree:} Transduce a dependency tree to target graph \\
%An event for each head. Each dependent is linked to the event via its dependency relation. %Head is linked to the event via \textit{arg0}.

\vspace{2em}
\hlight{\textsc{UDepLambda}:} Logical forms from Universal Dependencies \\

\end{frame}

\begin{frame}
\frametitle{Results on Multilingual WebQuestions}
\framesubtitle{English}
\centering
\large
\vspace{0.4em}
\includegraphics[trim=9em 0em 29em 1em,clip=true,scale=0.5]{figures/deplambda_results_plot_ud}
\end{frame}

\begin{frame}
\frametitle{Results on Multilingual WebQuestions}
\framesubtitle{English}
%\centering
\vspace{0.6em}
\only<1>{\includegraphics[trim=1em 12em 34em 1.5em,clip=true,scale=0.43]{figures/deplambda_results_plot}}
\only<2>{\includegraphics[trim=1em 12em 22em 1.5em,clip=true,scale=0.43]{figures/deplambda_results_plot}} 
\end{frame}

\begin{frame}
\frametitle{Results on Multilingual WebQuestions}
\framesubtitle{German}
\centering
\large
\vspace{0.4em}
\includegraphics[trim=9.5em 0em 23em 1em,clip=true,scale=0.5]{figures/deplambda_results_plot_ud-de}
\end{frame}

\begin{frame}
\frametitle{Results on Multilingual WebQuestions}
\framesubtitle{Spanish}
\centering
\large
\vspace{0.4em}
\includegraphics[trim=9.5em 0em 23em 0.5em,clip=true,scale=0.5]{figures/deplambda_results_plot_ud-es}
\end{frame}

\begin{frame}
\frametitle{Results on Multilingual GraphQuestions}
\framesubtitle{English}
%\centering
\large
\vspace{0.4em}
\only<1>{\includegraphics[trim=0em 0em 34em 0.2em,clip=true,scale=0.45]{figures/deplambda_results_plot_ud_gq_all}}
\only<2>{\includegraphics[trim=0em 0em 24em 0.2em,clip=true,scale=0.45]{figures/deplambda_results_plot_ud_gq_all}}
\end{frame}

\begin{frame}
\frametitle{Results on Multilingual GraphQuestions}
\framesubtitle{Spanish}
\centering
\large
\vspace{0.4em}
\includegraphics[trim=9.5em 0em 23em 0.5em,clip=true,scale=0.45]{figures/deplambda_results_plot_ud_gq-es}
\end{frame}

\begin{frame}
\frametitle{Results on Multilingual GraphQuestions}
\framesubtitle{German}
\centering
\large
\vspace{0.4em}
\includegraphics[trim=9.5em 0em 23em 0.5em,clip=true,scale=0.45]{figures/deplambda_results_plot_ud_gq-de}
\end{frame}

\begin{frame}
\frametitle{Error Analysis / Limitations}
\large
Context-sensitive semantics of dependency labels, e.g., \textit{nsubj} is \textbf{not} always agent (arg$_1$)
\begin{itemize}
\item John broke the window $\Large \textbf{\hlight{\cmark}}$
\vspace{1em}
\item The {\color{red} window} broke $\Large \textbf{\color{red} {\xmark}}$
\begin{itemize}
\item \large \textsl{window} is the patient (arg$_2$) although it occurs as \textsl{nsubj}
\end{itemize}
\end{itemize}

\vspace{2em}
\hlight{Solution:} Semantic Role labeling? 
\end{frame}

\begin{frame}
\frametitle{Results on Multilingual WebQuestions}
\framesubtitle{English}
\centering
\large
\vspace{0.4em}
\includegraphics[trim=9em 0em 22em 1em,clip=true,scale=0.45]{figures/deplambda_results_plot_ud}
\end{frame}

\section{Summary}
\begin{frame}
\frametitle{Summary}
\large
 
\vspace{1em} 
Language-agnostic method for converting \\Universal Dependencies to Logical forms

\vspace{2em}
New Freebase evaluation datasets in German and Spanish

\vspace{2em}
Ongoing Work: Richer Type System and Scoped Semantics

\vspace{2em}
Code: \hlight{\url{github.com/sivareddyg/UDepLambda}} \\
Demo: \hlight{\url{sivareddy.in/udeplambda.html}} \\

\begin{center}
 \large \hlight{Thank You!}
\end{center}
\end{frame}


\section{Additional Slides}

\begin{frame}
\frametitle{Quantifiers and Negation Scope}
\framesubtitle{(Fancellu et al. 2017, Reddy et al. 2017)}
\Large 
Higher-order type system \\
\vspace{2em}
Fine-grained dependency labels
\end{frame}


\begin{frame}
\frametitle{Quantifiers and Negation Scope}
\framesubtitle{Fancellu et al. 2017, Reddy et al. 2017}
\centering

\includegraphics[trim=0em 0em 0em 0em,clip=true,scale=1]{figures/everybody-wants-buy-house-crop} \\

\tikz[remember picture] \node[coordinate] (n1) {};\\
\vspace{1.5cm} 
\tikz[remember picture] \node[coordinate] (n2) {};

\begin{tikzpicture}[remember picture,overlay]   %% use here too
  \draw [->,>=stealth,shorten >=1pt,blue!20!white,line width=10pt] (n1) -- node[xshift=6em,black] { } (n2);
  \draw [->,>=stealth,blue!20!white,sloped,line width=10pt] (n1) -- node[black] {\scriptsize Enhancement} (n2);
\end{tikzpicture}

\includegraphics[trim=0em 0em 0em 0em,clip=true,scale=1]{figures/everybody-wants-buy-house-enhanced-with-univ-crop}

\end{frame}


\begin{frame}
\frametitle{Quantifiers and Negation Scope}
\centering
\includegraphics[trim=0em 0em 0em 0em,clip=true,scale=1.1]{figures/everybody-wants-buy-house-enhanced-with-univ-crop}

\begin{block}{\large Type System}
\large
\begin{tabbing}
everybody \= =  \= $\lambda x. \mathrm{everybody}(x_a)$ \hspace{2.8em} \=  [Old Type]\\
\> = \> $\lambda f \lspace \forall x \lspace \mathrm{person}(x) \rightarrow f(x)$ \> [New Type]  \\

\\

wants \> = \> $\lambda x  \lspace \mathrm{wants}(x_e) $  \> [Old Type] \\
\> = \> $\lambda f \lspace \exists x \lspace \mathrm{wants}(x_e) \land f(x)$ \> [New Type] 

\end{tabbing}
\end{block}
\end{frame} 

\begin{frame}
\frametitle{Quantifiers and Negation Scope}
\centering
\includegraphics[trim=0em 0em 0em 0em,clip=true,scale=1.1]{figures/everybody-wants-buy-house-enhanced-with-univ-crop}

\begin{block}{\large Type System}
\large
\begin{tabbing}
nsubj \hspace{2em} \= =  \= $\lambda f g x\lspace \exists y\lspace f(x) \land g(y) \land \mathrm{arg_1}(x_e, y_a)$ \hspace{2.6em} \=  [Old]\\
nsubj:univ \> = \> $\lambda P Q\, f \lspace Q(\lambda y \lspace P (\lambda x \lspace f(x)  \land \mathrm{arg_1}(x_e,y_a)))$ \> [New]  \\

\\

dobj \> = \> $\lambda f g x\lspace \exists y\lspace f(x) \land g(y) \land \mathrm{arg_2}(x_e, y_a)$ \> [Old] \\
\> = \> $ \lambda P\, Q\, f \lspace P(\lambda x \lspace f(x) \land  Q(\lambda y \lspace \mathrm{arg_2}(x_e,y_a))) $ \> [New] 

\end{tabbing}
\end{block}
\end{frame} 

\begin{frame}
\frametitle{Quantifiers and Negation Scope}
\centering
\includegraphics[trim=0em 0em 0em 0em,clip=true,scale=1.1]{figures/everybody-wants-buy-house-enhanced-with-univ-crop}

\raggedright \textbf{Old Expression:}
\begin{tabbing}
(3) $\lambda z \lspace \exists xyw \lspace$\=$\mathrm{wants}(z_e) \land \mathrm{everybody}(x_a) \land \mathrm{arg_1}(z_e, x_a) $\\
\>$\land\; \mathrm{buy}(y_e) \land \mathrm{xcomp}(z_e, y_e) \land \mathrm{arg_1}(y_e, x_a)$\\
\>$\land \; \mathrm{arg_1}(x_e,y_a) \land \mathrm{house}(w_a) \land \mathrm{arg_2}(y_e, w_a)$\,.
\end{tabbing}

\raggedright \textbf{New Expression:}
\begin{tabbing}
(6) $\lambda f \lspace \forall x$ \= $\lspace \mathrm{person}(x_a) \rightarrow$\\\
\>$[\exists zyw \lspace$\=$f(z) \land \mathrm{wants}(z_e) \land \mathrm{arg_1}(z_e, x_a)\land \mathrm{buy}(y_e)$\\
\>\>$\land\; \mathrm{xcomp}(z_e, y_e) \land\, \mathrm{house}(w_a)$\\
\>\>$\land\; \mathrm{arg_1}(z_e, x_a) \land \mathrm{arg_2}(z_e, w_a)]$\,.
\end{tabbing}

\end{frame} 



\subsection{Single Type System}

\begin{frame}[noframenumbering]
\frametitle{Single Type System}
\vspace{-3em}
\begin{center}
\includegraphics[trim=2em 9.4em 10em 0em,clip=true,scale=1.3]{figures/pixar_dobj}

%\vspace{1cm}

%{({\color{red} dobj} {\color{blue} acquired} {\color{blue!40!green!60!black} Pixar})}
\end{center}

\vspace{1cm}

\begin{block}{\centering All constituents are of the same lambda expression type}
\centering
\vspace{0.1cm}
\type{{\color{blue} acquired}} =  \type{{\color{blue!40!green!60!black} Pixar}}  = \type{{({\color{red} dobj} {\color{blue} acquired} {\color{blue!40!green!60!black} Pixar})}}
\vspace{0.1cm}
\end{block}
\end{frame}

\begin{frame}[noframenumbering]
\frametitle{Single Type System}
\vspace{-0.8em}
\begin{center}
\includegraphics[trim=2em 9.4em 10em 0em,clip=true,scale=1.3]{figures/pixar_dobj}

%\vspace{1cm}

%{({\color{red} dobj} {\color{blue} acquired} {\color{blue!40!green!60!black} Pixar})}
\end{center}

\vspace{1cm}

\begin{block}{\only<1>{All \textbf{words} have a \textit{lambda expression} of type $\eta$}
\only<2->{All \textbf{constituents} have a \textit{lambda expression} of type $\eta$}}
\vspace{0.5em}
\begin{itemize}
  \item \type{{\color{blue} acquired}} = $\eta$
  \vspace{0.5em}
  \item \type{{\color{blue!40!green!60!black} Pixar}} = $\eta$
  \vspace{0.5em}
  \item<2-> \type{{({\color{red} dobj} {\color{blue} acquired} {\color{blue!40!green!60!black} Pixar})}} = $\eta$
  \vspace{1em}
\end{itemize}
\vspace{-0.3cm}
\visible<3->{$\implies$ \type{{\color{red} dobj}} = $\eta \rightarrow \eta \rightarrow \eta$}
\end{block}
\end{frame}

\begin{frame}[noframenumbering]
\frametitle{Single Type System}
\vspace{-2.4em}
\begin{center}
\includegraphics[trim=2em 9.4em 10em 0em,clip=true,scale=1.3]{figures/pixar_dobj}

\end{center}

\vspace{1cm}

\begin{block}{Lambda Expression for words}
\vspace{-0.5cm}
\begin{align*}
  \hbox{\color{blue} acquired} & \Rightarrow  \lx_e \lspace \hbox{acquired}(x_e) & \visible<2->{\Rightarrow & \textbf{\textsc{type}} = \etype \rightarrow \btype} \\
  \hbox{\color{blue!40!green!60!black} Pixar} & \Rightarrow  \lx_a \lspace \hbox{Pixar}(x_a) &  \visible<2->{\Rightarrow & \textbf{\textsc{type}} = \itype \rightarrow \btype}
\end{align*}
\vspace{-0.5cm}
\end{block}
\visible<2->{Here \type{\hlight{acquired}} $\color{red} \neq$ \type{{\color{blue!40!green!60!black} Pixar}}  \alert \xmark}
\end{frame}

\begin{frame}
\frametitle{Single Type System}
\vspace{-2.4em}
\begin{center}
\includegraphics[trim=2em 9.4em 10em 0em,clip=true,scale=1.3]{figures/pixar_dobj}

\end{center}

\vspace{1cm}

\begin{block}{\centering Lambda Expression for dependency labels}
\vspace{-0.5cm}
\begin{align*}
  \hbox{\alert{dobj}} & \Rightarrow  \lambda \hlight{\mathbf{f}}\;\; \lambda {\color{blue!40!green!60!black} \mathbf{g}}\;\; \lambda \hlight{\mathbf{z}}\; \lspace \; \exists {\color{blue!40!green!60!black} \mathbf{x}}\; \lspace \; \hlight{\mathbf{f(z)}}\;\;\; \land \;\;\; {\color{blue!40!green!60!black} \mathbf{g(x)}}  \;\;\; \land\;\;\; \alert{\mathbf{arg_2}}(\hlight{\mathbf{z_e}}, {\color{blue!40!green!60!black} \mathbf{x_a}})
\end{align*}
\vspace{-0.5cm}
\end{block}
\visible<2->{\hspace{3cm} This operation mirrors the tree structure}
\end{frame}

\begin{frame}[noframenumbering]
\frametitle{Single Type System}
\vspace{-2.4em}
\begin{center}
\includegraphics[trim=2em 9.4em 10em 0em,clip=true,scale=1.3]{figures/pixar_dobj}
\end{center}

\vspace{1cm}

\begin{block}{Lambda Expression for words}
\vspace{-0.5cm}
\begin{align*}
  \hbox{\color{blue} acquired} & \Rightarrow  \lambda \mathbf{\color{red} x_a} x_e \lspace \hbox{acquired}(x_e) & \visible<2->{\Rightarrow & \textbf{\textsc{type}} = \ptype  \rightarrow \btype} \\
  \hbox{\color{blue!40!green!60!black} Pixar} & \Rightarrow  \lx_a {\mathbf{\color{red} x_e}} \lspace \hbox{Pixar}(x_a) &  \visible<2->{\Rightarrow & \textbf{\textsc{type}} = \ptype \rightarrow \btype}
\end{align*}
\vspace{-0.5cm}
\end{block}
\visible<2->{Here $\eta$ = \type{\hlight{acquired}} = \type{{\color{blue!40!green!60!black} Pixar}} $\hlight \checkmark$} 
\end{frame}

\subsection{Complex constructions}

\begin{frame}[noframenumbering]
\frametitle{Conjunctions}

\textbf{Sentence:} \\
{\centering Eminem signed to Interscope and discovered 50 Cent. \\} 

\vspace{0.3cm}  
\textbf{Binarized tree:} \\
{\centering (nsubj (conj-vp (cc s\_to\_I and) d\_50) Eminem) \\}

\pause
\vspace{0.3cm}  
\textbf{Substitution:} 

\begin{tabbing}
  conj-vp $\Rightarrow$ $\lambda f g x \lspace \exists y z \lspace f(y) \land g(z) \land \mathrm{\hlight{coord}}(x, y, z)$
\end{tabbing}

\vspace{0.3cm}  
\textbf{Logical Expression:}
\begin{tabbing}
  $\lambda w \lspace$\=$\exists x y z \lspace \mathrm{Eminem}(x_a) \land \mathrm{\hlight{coord}}(w, y, z)$ \\
\> $\land\, \mathrm{arg_1}(w_e, x_a) \land\, \mathrm{s\_to\_I}(y) \land \mathrm{d\_50}(z)$
\end{tabbing}

\pause
\textbf{Post processing:}
\begin{tabbing}
$\lambda e \lspace$\=$\exists x y z \lspace \mathrm{Eminem}(x_a) \land \mathrm{arg_1}(y_e, x_a)$ \\
\> $\land\,\mathrm{arg_1}(z_e, x_a) \land \mathrm{s\_to\_I}(y) \land \mathrm{d\_50}(z)$
\end{tabbing}
\end{frame}

\subsection{Graph Operations}
\begin{frame}[noframenumbering]
\frametitle{Graph Transformation: \contract operation}

\begin{center}
  What language do the people in Ghana speak?
\end{center}

\begin{columns}
  \begin{column}{0.35\textwidth}
   \centering
   \includegraphics[width=\textwidth]{figures/en-language-ghana-ungrounded} \\
   \small Ungrounded graph
  \end{column}
  \begin{column}{0.35\textwidth}
    \centering
   \includegraphics[width=\textwidth]{figures/language-ghana-grounded} \\
   \small Grounded graph 
  \end{column}
\end{columns}
\end{frame}

\begin{frame}[noframenumbering]
\frametitle{Graph Mismatch: \expand operation}

 %\item Due to parse errors, sometimes not all entities are in the ungrounded graph
 %\item \expand operator connect these entities to the main event
 What to do Washington DC December? \\
 

 \vspace{0.5cm} 
 \textbf{Before \expand}
 \begin{itemize}
   \item $\lambda z \lspace \exists x y w \lspace \textsc{target}(x_a) \land
\mathrm{do}(z_e) \land \mathrm{arg_1}(z_e, x_a) \land
\mathrm{Washington\_DC}(y_a) \land \mathrm{December}(w_a)$
   \end{itemize}
 
 \vspace{0.5cm}
 \textbf{After \expand}
 \begin{itemize}
   \item $\lambda z\lspace \exists x y w \lspace \textsc{target}(x_a) \land \mathrm{do}(z_e) \land \mathrm{arg_1}(z_e, x_a) \land \mathrm{Washington\_DC}(y_a) \land
   \mathrm{\hlight{dep}}(z_e, y_a) \land \mathrm{December}(w_a) \land \mathrm{\hlight{dep}}(z_e, w_a)$
   \end{itemize}
\end{frame}

\end{document}
